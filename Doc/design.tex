% !TEX root = SystemTemplate.tex
\chapter{Design  and Implementation}
This section will describe the design details for each of the major components 
in the system. 
 

\section{Parse Directory Function}

\subsection{Technologies  Used}
C++, Linux.

\subsection{Component  Overview}
This function will look for a C++ file, once it finds a C++ file it will compile it. 
This function will then call the find test, and run difference functions to run a test
and to see if there is a difference between the actual and expected outputs.

\subsection{Phase Overview}
This can traverse a directory, it can find and compile source code. It can look for, find, 
and run test cases. It can check for differences, and write to a log file.



\section{Find Test Function}

\subsection{Technologies  Used}
This will use C++ and system call function

\subsection{Component  Overview}
This function will travese a directory to find, and run test cases in our compiled 
source code. It will also write if the test case passed or failed, and increment our 
total count of passed and failed for the end of the log file.

\subsection{Phase Overview}
This function can traverse a directory looking for test case.  It can also run those 
test cases when it finds them, and write the results to the log file.  It also increments
the counter. 


\section{Run Difference Function}

\subsection{Technologies  Used}
This uses the C++, and system call function, as well as the diff function in Linux/GNU

\subsection{Component  Overview}
This function will create a function that will run the diff function in Linux/GNU via the system call.
First however it must create the string that will invoke the call, and take in the file names. Then it 
will pass inform others if the test case passed or failed so they can handle the information accourdingly. 
Also this doesn't print anything to the screen if there is a difference between the two files.

\subsection{Phase Overview}
This function can create the command string that will compare two files. This function will call the diff 
function. This function will not output anything to the screen if there is a difference between the two files. 
This function informs if there is, or is not, a difference between the two files.



\section{Test Directory Crawl}

\subsection{Technologies Used}
This uses the C++ standard library namely the dirent.h library.

\subsection{Component Overview}
This function recusivly searches the test directory and upon finding a .tst file will run the function to test the
executable against it.  When the function finds a subdirectory it calls itself on that subdirectory, allowing it to
fully search for all of the test files.


\section{Student Directory Crawl}

\subsection {Technologies Used}
This uses the C++ standard library namely the dirent.h library.

\subsection{Component Overview}
This function creates a log file for the entire class in the root.  Then it searches for subdirectories other than the
Test directory, which contains only the .tst and .ans files.  When it finds an applicable subdirectory it changes in,
creates a log file for that student, and compiles the .cpp file found within.  Then it runs the test directory crawl
on the compiled program.  After it has returned from the test crawl it does the final log write for the student and
writes to the class log as well.


\section (Compile a Program}

\subsection {Technologies Used}
This uses the C++ standard library, system calls, and the g++ compiler.

\subsection{Component Overview}
This program recives a string which determines the name of the executable to be produced.  Then from the
current directory finds and compiles the first .cpp file found.


\section {Student Log Write}

\subsection {Technologies Used}
This uses the C++ standard library, namely the fstream library.

\subsection {Component Overview}
This function takes an ofstream, the name of the test, and the success or failure of that test, and writes the
formated results to the ofstream.


\section {Final Log Write}

\subsection {Technologies Used}
This uses the C++ standard library, namely the fstream library.

\subsection {Component Overview}
This function takes an ofstream, the name of the student, and a record of the pass and failure of all of the test cases.  
This function writes the percent of the tests passed, assuming that no critical tests were failed.  If a critical test was failed
then only "FAILED" is written, rather than the percentage.  This function is used to write the last line of each student log file,
as well as each line in the class log file.


\section {Run Test Case}

\subsection {Technologies Used}
This uses the C++ standard library and system calls to execute the program.

\subsection PComponent Overview}
This function takes the name of the .tst file and generates the name of the .ans and .out files.  The .out file being created in
the same directory as the one the .cpp file was found in, and the .ans in the same one as the .tst file was found in.
After the names are generated, using the system command the executable is run with the .tst file used as input
and the output being piped to the .out file.  Once it has completed the RunDiff function is called to determine
if the program executed properly.  The log file and record are updated accordingly.

